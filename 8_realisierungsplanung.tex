\chapter{Realisierungsplanung}
\label{cha:8}
% (ca. 1 1/2 Seiten)
%alle wichtigen Meilensteine definieren


%Meilensteine
Der Erwerb des Gebäudes mit einer Fläche von knapp 35.000 $m^{2}$ kostet ca. 300.000€.
Für das einrichten der Arbeitsbereiche sind zwei Wochen geplant In dieser Zeit werden schon intensive Werbemaßnahmen getroffen, um den Bekanntheitsgrad von Shozies zu erhöhen. Außerdem soll bis zum Produktionsbeginn die Website fertig sein und das Amazonkonto eingerichtet sein.
Der Lagerraum in Deutschland mit Büro soll ebenfalls bis dahin eingerichtet sein. Um von dort aus alles weitere geschäftliche regeln zu können. Auch die Entwicklung eines Datenbanksystems für die Kunden- und Artikeldaten muss bis dahin fertig sein.\\
Wir streben zunächst mindestens eine regionale Bekanntheit von Shozies GmbH an, die über Poster, Mundpropaganda und Facebook umgesetzt werden soll. Innerhalb von drei Jahren wollen wir deutschlandweit bekannt sein und innerhalb von zehn Jahren weltweit. Mit höherer Verbreitung unseres Produktes wollen wir entsprechend auch zusätzliche Produktionsgebäude erwerben, ausstatten und nutzen. Wir wollen innerhalb von fünf Jahren den aufgenommen Kredit vollständig zurückgezahlt haben.\\

%Cahncen
Im günstigsten Fall wird Shozies sehr schnell bekannt und ist sehr beliebt, sodass wir sehr hohe Verkaufszahlen erzielen. Die Herstellungsmaterial sind dauerhaft und günstig verfügbar. Dann wäre eine möglichst schnelle Erweiterung unserer Produktion möglich, um noch mehr Gewinn zu erzielen. Dann könnte die Abzahlung des Kredits evtl. auch schneller erfolgen. Allerdings müssten die Kosten für die eventuellen neuen Gebäude und für die größere Menge an Material berücksichtigt werden.

%Risiken
Risiken bestehen insofern, dass die Erhöhung des Bekanntheitsgrades oder die Beliebtheit des Produktes sehr schlecht laufen. Leute, die bereits von anderen Herstellern Zehenschuhe gekauft haben, könnten diesen Verkäufern treu bleiben. Sollten unsere Produkte zu Beginn des Verkaufs aufgrund von qualitativen Mängeln schlechte Rezensionen erhalten, würde sich dies sehr negativ auf die Verkaufszahlen auswirken. Sollten zu viele Kunden die Shozies aus irgendwelchen Gründen zurückschicken, kann es passieren, dass wir zu viele gelagerte Produkte habe, die niemand kauft. Sollten Umwelteinflüsse oder andere Gründe dafür sorgen, dass unser Material nicht oder nur sehr teuer verfügbar ist, führt dies zu einem niedrigeren Gewinn für uns oder zu einer Preiserhöhung unseres Produktes, wodurch wir evtl. Käufer verlieren könnten.