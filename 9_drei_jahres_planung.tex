\chapter{Drei-Jahres-Planung}
\label{cha:9}
% (ca. 2 Seiten)
%1.Jahr
Zur Ermittlung unserer Einnahmen im ersten Jahr sind wir davon ausgegangen, dass wir pro Schuh durchschnittlich ca. 80€ Umsatz machen. Der Verkaufspreis liegt bei durchschnittlich 200€ inkl. Mehrwertsteuer von 38€ und der Materialkosten von etwa 80€.
Wenn wir pro Monat 300 Paar Schuhe verkaufen kommen wir im ersten Jahr auf 3600 verkaufte Paar Schuhe und einen Umsatz von 288.000€. Dazu kommen die Einnahmen, die wir durch die Pflegemittel erhalten. Wir gehen davon aus das bei 20\% der Einkäufe ein Pflegemittel dazu gekauft wird, welches uns einen durchschnittlichen Gewinn von 10€ einbringt. Dies würde einen Umsatz von 7.200€ im ersten Jahr bringen.
Wir kämen so auf einen Gesamtumsatz von 295.200€ im ersten Jahr.\\
Die Personalkosten werden bei auf ca. 80.000€ und die Fahrzeugkosten (inklusive Sprit und Wartung) bei ca. 70.000€ im ersten Jahr liegen. Hinzu kommen die Nebenkosten der Gebäude von ca. 50.000€ jährlich.\\
%2. Jahr
Im zweiten Jahr sollen monatlich 500 Paar Schuhe verkauft werden, sodass wir durch den Verkauf der Schuhe und Pflegemittel einen Umsatz von 492.000€ erzielen.
Die Personalkosten werden auf 135.000€ steigen.\\
%3.Jahr
Ab dem dritten Jahr sollen monatlich 800 Paar Schuhe verkauft werden, daraus folgt ein Umsatz von 787.200€. Die Personalkosten steigen dann auf ca. 215.000€ im Jahr.\\
