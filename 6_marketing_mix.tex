\chapter{Marketing-Mix}
\label{cha:6}
% (ca. 1 1/2 Seiten)

\section{Werbemaßnahmen}
Werbemaßnahmen sind für unser Unternehmen besonders wichtig, damit potentielle Kunden von uns erfahren. Deshalb planen wir, unsere Shozies auf unterschiedlichsten Wegen bekannt zu machen. Wir wollen die Shozies zum einen auf verschiedenen Onlineplattformen wie z.B. Amazon oder ebay verkaufen und in Sozialnetzwerken wie facebook.com, twitter.com, instagramm.com durch posts, tweets und Bilder Werbung machen, um die Kunden auf die Shozies aufmerksam zu machen. Zudem werden wir eine eigene Homepage online stellen, Flyer verteilen und Poster aufhängen. 

Die Werbung wollen wir so gestalten, dass wir vom Kunden als gesunde, edle, gemütliche und ausgefallene Alternative wahrgenommen werden. Zudem locken wir die Kunden mit Zusatzleistungen zum Produkt, wie Pflegemittel, maßgeschneiderte Schuhe, oder spezielle Sohlen. Außerdem sollen produktbezogene Kundenwünsche geäußert werden können. Diese werden wir so gut wir möglich umsetzen. Ein weiteres Angebot ist dieses: Wir bieten gute Qualität für einen guten Preis. Nach dem Kauf von insgesamt sechs Paar Schuhen erhalten die Kunden jeweils einen 10\% Gutschein. Beim Kauf von drei Paar Schuhen auf einmal erhalten die Kunden 10\% Rabatt auf den Einkauf. Den Vertrieb der Schuhe übernehmen wir selbst.

Der große Vorteil unseres Produkt ist der, dass er nicht erklärt werden muss. Anhand des Namens kann man darauf schließen, dass es sich um Schuhe handelt und anhand der Werbung wird man sehr schnell dem Begriff \glqq Shozie\grqq\ ein Bild zuordnen können. 

\section{Preise}
Das Ziel ist es, die Schuhe für den Kunden zu einem angemessenen Preis zu verkaufen. Nach unseren Recherchen bräuchten wir für das Leder eines Schuhs ca. 55€. Die Sohle würde zwischen 10€ und 20€ kosten. Rechnet man noch den Versand und die Bezahlung der Arbeitskräfte hinzu, wird ein Schuh je nach Gestaltung und Verarbeitung zwischen 150€ und 220€ kosten.
